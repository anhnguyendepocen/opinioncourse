\documentclass[12pt,a4paper]{article}
\usepackage[margin=1in]{geometry}
\usepackage{graphicx,setspace,hyperref,amsmath,amsfonts,multirow,ccaption,mdwlist,comment}

\usepackage{minitoc}
\dosecttoc
\setcounter{secttocdepth}{2}
\renewcommand{\stctitle}{}
\nostcpagenumbers

\setlength{\marginparwidth}{.5in}
\usepackage{natbib}
\usepackage{bibentry}
\newcommand{\reading}[2][]{\noindent -- {#1 }\bibentry{#2}.\vspace{.25em}\\}
\newcommand{\seealso}{\noindent \emph{See Also:}\\}
\newcommand{\topic}[1]{\noindent \textbf{#1}\\}
\usepackage[T1]{fontenc}
\usepackage{lmodern}
\hypersetup{
    bookmarks=true,         % show bookmarks bar?
    unicode=false,          % non-Latin characters in Acrobat’s bookmarks
    pdftoolbar=true,        % show Acrobat’s toolbar?
    pdfmenubar=true,        % show Acrobat’s menu?
    pdffitwindow=false,     % window fit to page when opened
    pdfstartview={FitH},    % fits the width of the page to the window
    pdftitle={Syllabus: Public Opinion, Political Psychology, and Citizenship},    % title
    pdfauthor={Thomas J. Leeper},     % author
    pdfsubject={Political Science},   % subject of the document
    pdfkeywords={politics} {public opinion} {political psychology}, % list of keywords
    pdfnewwindow=true,      % links in new window
    pdfborder={0 0 0}
}

\begin{document}
\nobibliography*
\faketableofcontents

\begin{center}
{\Large
\noindent \textbf{GV4J3\\ Public Opinion, Political Psychology, and Citizenship}
}
\end{center}
\vspace{1em}

\noindent
Dr Thomas J. Leeper\\
Office: CON 3.21\\
Office hours: By appointment via LfY\\
Email: \href{mailto:t.leeper@lse.ac.uk}{t.leeper@lse.ac.uk}

\vspace{1em}

\noindent Course website:\\ \url{https://moodle.lse.ac.uk/course/view.php?id=5109}\\
Reading list:\\ \url{https://library-2.lse.ac.uk/e-lib/e_course_packs/GV4J3/GV4J3_64769.pdf}\\

\noindent The purpose of this course is to explore issues related to public opinion, including what opinions are and how they are formed, what factors do and do not influence opinion development and change, how opinions drive citizens' political thinking and behaviour, and what implications these psychological processes have for the role of public opinions in democratic government. Students will leave the course with a thorough theoretical understanding of political opinions, their origins, and their possible effects through exposure to philosophical perspectives, contemporary case studies, and a broad set of empirical research.

\section{Objectives and Evaluation}
After this course, students should be able to:
\begin{enumerate*}
\item Explain what opinions are and how they are formed.
\item Describe properties of public opinion at the individual and aggregate levels.
\item Evaluate political psychological theories and normative arguments about public opinion.
\item Evaluate the quality of empirical public opinion research.
\item Explain and apply quantitative and qualitative methods to the study of public opinion.
\end{enumerate*}

\noindent It is important to note that this is a research seminar that is informed by original political science research and evaluated through participants' own original research paper. It may be useful to think of the course as a ``mini-dissertation'' project.

\section{Summative Assessment: Exam Paper}

The exam for the course is an independent research paper of approximately 5,000 words that:

\begin{enumerate}
\item addresses an important political science question related to public opinion, political psychology, or political behaviour,
\item offers a theoretical contribution toward understanding that question, and 
\item reports an original empirical analysis that tests that theory.
\end{enumerate}

Original data analysis (and possibly data collection) are required, though the particular form of the empirical component can be qualitative, quantitative, or both. Some examples of empirical projects include: an original survey and/or experimental data collection, a pilot test of a proposed research design (and the description of a more complete empirical design), qualitative data analysis (such as focus groups, semi-structured interviewing, content analysis, etc.), the analysis of existing public opinion data (e.g., surveys, cross-national comparisons, election results, etc.), or some mix of these. While it is not expected that students conduct a large-scale study, they must conduct some novel data collection and/or analysis.

Students pursuing original data collection (of any kind) must comply with the LSE Research Ethics Policy\footnote{\url{http://www.lse.ac.uk/intranet/researchAndDevelopment/researchDivision/policyAndEthics/ethicsGuidanceAndForms.aspx}} and complete an ethical self-assessment form (to be signed by the instructor) before gathering any data.

The exam essay will be marked according to guidelines available in \href{http://www.lse.ac.uk/government/degreeProgrammes/programmes/masters/PDF/MSc-Student-Handbook.pdf}{the Government Department MSc Handbook}. Marks are assigned according to the conventional LSE scale and written feedback will be provided on the assessed essay.

\vspace{1em}

\noindent The essay is due \textbf{Tuesday 25 April 2017 at 5:00pm}.


\section{Formative Activities and Assessment}

Formative assessment consists of (1) a 2-page written proposal for the final essay and an associated literature review, (2) four out-of-class problem sets submitted near the beginning of the term, and (3) in-class discussion activities.

\subsection{Research Proposal}

In preparation for the final exam, students will prepare a short, 2-page proposal to be submitted in Week 6 of Lent Term that outlines a possible project for the assessed essay. This document should state a research topic and clear research question, make reference to relevant theoretical and empirical literature, and propose a basic design for addressing the question. Students should focus on one topic, but can present up to two distinct ideas if they are undecided about what to do. The proposal should be uploaded to Moodle by the beginning of Week 6.

Students should then meet one-on-one with the instructor during Week 6 to discuss the ideas, receive feedback, and make plans for the final paper. Once a topic is agreed, student should use Week 6 to complete an annotated bibliography of 5--10 relevant studies (from reading list and elsewhere) that motivate the final project and upload it to Moodle.

Once finalized, 3--4 students per week will be asked to briefly present their projects for peer feedback during class meetings in Weeks 7--10 of term. These presentations should be oral and last about 5 minutes.

\subsection{Problem Sets}

Given the combination of an assessed essay as the sole summative assessment, a relatively short term (10 weeks), and the varied backgrounds of students enrolled in the course, short problem sets applying different research methods in public opinion are due in the first four weeks of the course (Weeks 2--5). These provide an opportunity to both gain methodological competence to critique readings in the course and prepare the final exam project.

\begin{center}
\begin{tabular}{ll} \hline
\textbf{Problem Set} & \textbf{Due Date} \\ \hline
Week 1: Interviewing & January 17 \\
Week 2: Trends and toplines & January 24 \\
Week 3: Correlation and regression & January 31 \\
Week 4: Experimentation & February 7 \\ \hline
\end{tabular}
\end{center}

The problem sets take the form of ``replication'' activities, in which the data from a published research article is made available and students are asked to reproduce the results of the paper from the original data and explain the logical of the underlying methods. The problem sets are mandatory but are not marked. Please treat them as an opportunity to self-evaluate and learn and to approach the instructor with any hesitations you may have. Collaboration is allowed, but each student should submit an individual assignment. Marking rubrics will be provided.

\subsection{Discussion Activities}

The course will primarily involve student-led discussions with the exception of a few lecture elements surrounding methodological issues in public opinion research. The course is structured as a ``reading group,'' where every student is expected to have read all assigned readings and should be able to summarize and critique each reading if asked to do so.

In preparing for discussion students should be able to summarize and critique several key parts of each article:

\begin{enumerate}
\item What is the research question?
\item What is the theory? Is it clearly argued and reasonable?
\item To which literature does the article contribute?
\item What are the hypotheses or expectations? Do these derive clearly from theory? Are they falsifiable?
\item What is the method of analysis? How are data collected? How appropriate are the method and data?
\item What are the results? Do they support the proposed theory?
\end{enumerate}

\noindent It may be useful to write out answer to each of these questions for every article.

Additionally, given the discussion format of the course, every week, every student must post 1 or 2 discussion questions to Moodle based upon the topic of the week and the assigned readings. These are due by Thursday at 17:00 (5:00pm) prior to class.



\clearpage
\section{Course Outline}

Class will meet at the following times and locations:

\begin{itemize}
\item Group 1: Friday 15:00-17:00 (NAB.1.18) in LT Weeks 1--5,7--11
\item Group 2: Friday 13:00-15:00 (NAB.2.13) in LT Weeks 1--5,7--11
\end{itemize}


\noindent The course does not meet during reading week (LT Week 6). The general schedule for the course is as follows. Details on the readings for each week are provided on the following pages.

\secttoc


\clearpage
\subsection{Week 1: Conceptualizations of ``Public'' ``Opinion'' (Jan. 13)}

\noindent Last part of class will focus on material covered by Problem Set 1, which is due on Moodle by the beginning of Week 2.\\

\reading[Chapters 3\footnote{ \url{https://contentstore.cla.co.uk/secure/link?id=24aad0aa-b22c-e611-80bd-0cc47a6bddeb}} from ]{Herbst1995}
\reading{Mansbridge2003}
\reading{Disch2011}
\reading{ConoverSearingCrewe2008}

\seealso
\reading{UrbinatiWarren2008}
\reading{Druckman2014}
\reading{Dahl1989}
\reading{Riker1988}
\reading{Lippmann1922}
\reading{Lippmann1928}
\reading{Dahl2006}
\reading{Pitkin1967}
\reading{Downs1957}
\reading{HibbingTheiss-Morse2002}
\reading{JacobsPage2005}
\reading{BachrachBaratz1962}
\reading{Fishkin2006}
\reading[Chapter 8 from]{Schattschneider1975}
\reading{McCloskyHoffmanOHara1960}
\reading{Erikson1978}


\subsection{Week 2: Voting Behaviour (Jan. 20)}

\noindent First part of class will focus on material covered by Problem Set 2, which is due on Moodle by the beginning of Week 3.\\

\reading{McGrawLodgeStroh1990}
\reading{GreenHobolt2008}
\reading{UtychKam2014}
\reading{LauPatelFahmyKaufman2014}

\seealso
\reading{IyengarLepper2000}
\reading{Berelsonetal1954}
\reading{Key1966}
\reading{CarminesStimson1980}
\reading{Bartels1986}
\reading{RabinowitzMacDonald1989}
\reading{LauRedlawsk1997}
\reading{Brooks2011a}
\reading{Hobolt2005}
\reading{Jacobson2015}
\reading{RudolphPopp2013}
\reading{Wellsetal2009}
\reading{Selbetal2009}


\subsection{Week 3: What are attitudes? (Jan. 27)}

\noindent First part of class will focus on material covered by Problem Set 3, which is due on Moodle by the beginning of Week 4.\\


\reading[Chapter 7\footnote{\url{https://contentstore.cla.co.uk/secure/link?id=25aad0aa-b22c-e611-80bd-0cc47a6bddeb}} from ]{EaglyChaiken1998}
\reading{EaglyChaiken2007}
\reading{JohnstonWronski2013}

\seealso
\reading{DruckmanLupia2000}
\reading{EaglyChaiken1993}
\reading{Fazio2007}
\reading{Ajzen2001}
\reading{PettyKrosnick1995}
\reading{Hovland1953}
\reading{McGuire1969}
\reading{Katz1960}
\reading{Zaller1992}



\subsection{Week 4: Media and Social Influence (Feb. 3)}

\noindent First part of class will focus on material covered by Problem Set 4, which is due on Moodle by the beginning of Week 5.\\


\reading{FraileIyengar2014}
\reading{NelsonOxleyClawson1997}
\reading{Feldman2011a}
\reading{Klar2014a}

\seealso
\reading{MutzMartin2001}
\reading{Mutz2002}
\reading{DruckmanNelson2003}
\reading{MillerKrosnick2000}
\reading{LechelerdeVreese2011}
\reading{Gross2008}
\reading{Boczkowskietal2011}
\reading{EriksonStoker2011}
\reading{Mutz1992}
\reading{Hartetal2009}
\reading{Stroud2010}
\reading{DellaVignaKaplan2007}
\reading{AlbertsonLawrence2009}
\reading{McPhersonSmithLovinCook2001}
\reading{HuckfeldtSprague2006}
\reading{Nickerson2008}
\reading{BennettIyengar2008}
\reading{ChongDruckman2007b}
\reading{DruckmanFeinLeeper2012}
\reading{Gerberetal2011}


\subsection{Week 5: Motivated Reasoning (Feb. 10)}

\noindent Note: Research proposal is due to Moodle by Monday of Week 6. Please book a one-on-one meeting with the instructor via LSE for You.\\

\reading{Kunda1990}
\reading{ReedAspinwall1998}
\reading{TetlockSkitkaBoettger1989}
\reading{ChenSchecterChaiken1996}

\seealso
\reading{Nir2011}
\reading{TaberLodge2006}
\reading{DruckmanBolsen2011}
\reading{SlothuusdeVreese2010}
\reading{LeeperSlothuus2014}
\reading{MoldenHiggins2004}
\reading{RedlawskCivettiniEmmerson2010}
\reading{Druckman2012}
\reading{DruckmanPetersonSlothuus2013}
\reading{BolsenDruckmanCook2013}
\reading{HartNisbet2011}
\reading{Groenendyk2013}
\reading{Cohenetal2007}
\reading{KruglanskiWebster1996}
\reading{Dittoetal1998}
\reading{RedlawskCivettiniEmmerson2010}



\subsection{Week 6: Reading Week (Feb. 17) -- no class meeting}

There will be no class meeting on February 19 (LT Week 6) due to LT Reading Week. By this point in the course, students should have an idea in mind for their final exam essay topic and should have uploaded a 2-page proposal to Moodle by the end of Week 5. Please schedule a meeting with the instructor during Week 6 to discuss your proposal and then use the remainder of the week to prepare an annotated bibliography or literature review related to your proposal.


\subsection{Week 7: Attitude Strength and Attitude Change (Feb. 24)}

\reading{VisserBizerKrosnick2006}
\reading{Wood2000}
\reading{PomerantzChaikenTordesillas1995}

\seealso
\reading{ChongDruckman2007a}
\reading{PettyCacioppo1986}
\reading{Krosnick1988b}
\reading{AlbarracinMitchell2004}
\reading{VisserMirabile2004}
\reading{Holbrooketal2005}
\reading{Barabasetal2015}
\reading{DelliCarpiniKeeter1997}
\reading{LodgeMcGraw1995}
\reading{LodgeSteenbergenBrau1995}
\reading{Iyengar1990}
\reading{LauRedlawsk2001}
\reading{GronlundMilner2006}
\reading{TilleyWlezien2008}
\reading{Mutz1992}
\reading{LaxPhillips2009a}
\reading{DruckmanLeeper2012b}
\reading{PageShapiroDempsey1987}
\reading{MacKuenEriksonStimson1989}
\reading{Wlezien2012}
\reading{DruckmanLeeper2012a}
\reading{Gilens2001}
\reading{Sanders2012}
\reading{PageShapiro1992}
\reading{MulliganGrantBennett2013}
\reading{Knobloch-Westerwick2012a}
\reading{Wojcieszak2011a}


\subsection{Week 8: Political Identity, Values, and Other Predispositions (Mar. 3)}

\reading{AlfordFunkHibbing2005}
\reading{SearsFunk1991}
\reading{Huddy2001}
\reading{WhiteLairdAllen2014}

\seealso
\reading{Gay2004}
\reading{Klar2013}
\reading{Greene1999}
\reading{SuhayJayaratne2012}
\reading{Jackson2011a}
\reading{KangBodenhausen2015}
\reading{HuddyMasonAaroe2015}
\reading{KnowlesGardner2008}
\reading{KangBodenhausen2015}
\reading{WhiteLairdAllen2014}
\reading{GreenPalmquist1994}
\reading{Gerberetal2011b}
\reading{JostFedericoNapier2009}
\reading{Feldman1988}
\reading{TreierHillygus2009}
\reading{BrewerGross2005}
\reading{Converse1964}
\reading{JenningsNiemi1975}
\reading{Lacy2001a}
\reading{PeffleyHurwitz1985}
\reading{Federico2005}
\reading{CharneyEnglish2012}
\reading{LopezMcDermott2012}


\subsection{Week 9: Emotion and/or Cognition, Implicit and/or Explicit? (Mar. 10)}

\reading{LernerKeltner2000}
\reading{Huddyetal2005}
\reading{BraderValentinoSuhay2008}
\reading{Evans2008}

\seealso
\reading{Albertson2011}
\reading{Kam2007a}
\reading{ArceneauxVanderWielen2013}
\reading{IyengarSoodLelkes2012}
\reading{Mendelberg2010}
\reading{MarcusNeumanMacKuen2000}
\reading{BanksValentino2012}
\reading{LaddLenz2011}
\reading{MalhotraKuo2009}
\reading{Valentinoetal2008}
\reading{GrossDAmbrosio2004}
\reading{FishbachShahKruglanski2004}
\reading{Nabi1999}
\reading{Higgins1997}


\subsection{Week 10: Judgement and Decision-Making I (Mar. 17)}

\reading{Chaiken1980}
\reading{Hobolt2007}
\reading{MerollaStephensonZechmeister2008}
\reading{DvirGvirsman2015}


\seealso
\reading{Petersenetal2011}
\reading{MarksMiller1987}
\reading{Healy2010}
\reading{LauRedlawsk2001}
\reading{DanceySheagley2012}
\reading{TverskyKahneman1974}
\reading{TverskyKahneman1981}
\reading{Kahneman2012}
\reading{Ariely2008}
\reading{KuklinskiHurley1994}
\reading{Arceneaux2007}
\reading{ValentinoHutchingsWhite2002}
\reading{Lupu2013}

\subsection{Week 11: Judgement and Decision-Making II; Conclusion (Mar. 24)}

\reading{PetersenSznycerCosmidesTooby2012}
\reading{Kuklinskietal2000}
\reading{HealyMalhotraMo2010}
\reading{BottiIyengar2006}





\clearpage
\subsection{Appendix: Survey Research Methods}

Students analysing survey data may find some of the following readings useful.\\

\vspace{1em}

\reading{SingerCouper2014}
\reading{TourangeauRasinski1988}
\reading{BishopTuchfarberOldendick1986}
\reading{ZallerFeldman1992}
\reading{BanducciStevens2015}
\reading{Grovesetal2009}
\reading{Lohr2009}
\reading{Berinskyetal2011}
\reading{SchumanPresser1981}
\reading{SchaefferPresser2003}
\reading{KrosnickJuddWittenbrink2005}
\reading{RevillaSarisKrosnick2013}
\reading{YeagerLarsonKrosnickTompson2010}
\reading{Krosnicketal2002}
\reading{Brady2000}
\reading{Feldman1989}
\reading{AlvarezFranklin1994}
\reading{SchuldtKonrathSchwarz2011}
\reading{HillygusJacksonYoung2014}
\reading{Callegaroetal2014}
\reading{AndresGolschSchmidt2013}
\reading{AnsolabehereRoddenSnyder2008}
\reading{JohnstonBrady2002}
\reading{DelavandeManski2010}
\reading{GainesKuklinskiQuirk2007}


\bibliographystyle{plain}
\nobibliography{References}

\end{document}
