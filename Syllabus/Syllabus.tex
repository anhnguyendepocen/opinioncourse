\documentclass[12pt,a4paper]{article}
\usepackage[margin=1in]{geometry}
\usepackage{graphicx,setspace,hyperref,amsmath,amsfonts,multirow,ccaption,mdwlist,comment}
% mini table of contents
\usepackage{minitoc}
\dosecttoc % make section toc
\setcounter{secttocdepth}{2} % subsection depth
\renewcommand{\stctitle}{} % no title
\nostcpagenumbers

% optionally include commented environments
\excludecomment{lessonplan}
\excludecomment{finalexam}

\setlength{\marginparwidth}{.5in}
\usepackage{natbib}
% Two lines to create in-text full citations for a syllabus
% And comment out my other standard bibtext commands
\usepackage{bibentry}
\newcommand{\reading}[2][]{\noindent -- {#1 }\bibentry{#2}.\vspace{.25em}\\}
\newcommand{\seealso}{\noindent \emph{See Also:}\\}
\newcommand{\topic}[1]{\noindent \textbf{#1}\\}
\usepackage[T1]{fontenc}
\usepackage{lmodern}
\hypersetup{
    bookmarks=true,         % show bookmarks bar?
    unicode=false,          % non-Latin characters in Acrobat’s bookmarks
    pdftoolbar=true,        % show Acrobat’s toolbar?
    pdfmenubar=true,        % show Acrobat’s menu?
    pdffitwindow=false,     % window fit to page when opened
    pdfstartview={FitH},    % fits the width of the page to the window
    pdftitle={Syllabus: Public Opinion, Political Psychology, and Citizenship},    % title
    pdfauthor={Thomas J. Leeper},     % author
    pdfsubject={Political Science},   % subject of the document
    pdfkeywords={politics} {public opinion} {political psychology}, % list of keywords
    pdfnewwindow=true,      % links in new window
    pdfborder={0 0 0}
}

\title{Public Opinion, Political Psychology, and Citizenship}
\author{Thomas J. Leeper\\
Department of Government\\
London School of Economics and Political Science}

\begin{document}
\nobibliography*

\maketitle

\faketableofcontents

\section{Overview}

The purpose of this course is to explore issues related to public opinion, including what opinions are and how they are formed, what factors do and do not influence opinion development and change, how opinions drive citizens' political thinking and behaviour, and what implications these psychological processes have for the role of public opinions in democratic government. Students will leave the course with a thorough theoretical understanding of political opinions, their origins, and their possible effects through exposure to philosophical perspectives, contemporary case studies, and a broad set of empirical research. The course will challenge assumptions about what democracy is and how it works, explore what it means to be a good citizen in a contemporary democracy, and provide students with insight into how democratic governments can and should respond to the public's views. The focus will be on how citizens form political opinions, think and reason about policy debates, and act on their opinions, especially outside of elections, across a broad array of country contexts.

\section{Objectives and Evaluation}
After this course, students should be able to:
\begin{enumerate*}
\item Explain what opinions are and how they are formed.
\item Describe properties of public opinion at the individual and aggregate levels.
\item Explain different conceptualizations of political representation and their empirical implications.
\item Evaluate arguments about the proper role of public opinion in democracy and government.
\item Apply knowledge of opinions and opinion measurement to the evaluation of survey public opinion research.
\end{enumerate*}


\section{Formative Assessment

\subsection{Discussion Activities}

The course will primarily involve student-led discussions with the exception of a few lecture elements surrounding methodological issues in public opinion research.

Given the discussion format of the course, some preparation is required on the part of each student for each seminar meeting:

\begin{itemize}
\item Every week, every student must post 1 or 2 discussion questions to Moodle based upon the topic of the week and the assigned readings. These are due by Thursday at 17:00 (5:00pm) prior to class.
\item Each student will for one week's meeting serve as discussion co-leader (there will be 2-3 discussion leaders per week). This requires them to review the discussion questions posted to Moodle and facilitate a conversation around the assigned readings. It also requires a short writing task, described next, which 
\end{itemize}

Each discussion leader should write a short (2-3 page; max 1500 words) reflection paper that synthesizes course readings for that week, raises questions unanswered by the texts, or proposes avenues for further theoretical development and/or empirical research. These should be opinionated essays that make an argument about public opinion; they should not be simply summaries of the readings.

Reflection papers should be shared with the instructor and classmates via Moodle by Friday at 12:00 (noon) with the expectation that fellow students provide peer feedback during the seminar. Students will also receive written feedback from the instructor. These papers will help students to formulate and consider potential research topics for their final paper.

Class will start each week with these students discussing their papers. All students should read the papers before class.

\subsection{Problem Sets}

There are three problem sets for the course, which are designed to assess students' familiarity and competence with the basic (quantitative) methodological tools of public opinion research: experimental design and analysis, collection and analysis of original survey data, and the reanalysis of already collected survey (or survey-experimental) data. These problem sets are intended to prepare students for the methodological aspects of the exam.

\begin{center}
\begin{tabular}{lll} \hline
\textbf{Problem Set} & \textbf{Assigned} & \textbf{Due Date} \\ \hline
Micro Analysis & Jan. 22 & Jan. 26 \\
Macro Analysis & Jan. 29 & Feb. 2 \\
Experimentation & Feb. 5 & Feb. 9 \\ \hline
\end{tabular}
\end{center}

In each problem set, students will answer a series of short prompts about methods of public opinion research and provide simple analyses (using the statistical software of their choice) using data provided by the instructor. This is useful primarily for gauging students' competence with these methods, which may be necessary for the exam, and for the instructor to possibly prepare additional instruction on these topics as needed.


\section{Summative Assessment: Exam Paper}

The exam for the course is an independent research paper of approximately 5,000 words that addresses an important political science question related to public opinion, political psychology, or political behaviour, offers a theoretical contribution toward understanding that question, and reports a planned empirical analysis to test that theory. The requirements for the empirical component are flexible and it can involve an original survey and/or experimental data collection, a pilot test of a proposed research design, qualitative data analysis (such as focus groups or semi-structured interviewing), the analysis of existing public opinion data, or a mix of these. Thus, while it is not expected that students conduct a large-scale study, they must conduct some novel data collection and/or analysis (even at a very small scale).

The exam is due April 26, 2016 at 5pm.




\clearpage
\section{Course Outline}
The general schedule for the course is as follows. Details on the readings for each week are provided on the following pages.
\secttoc
\clearpage

% - Week 1 (Jan. 15): Public Opinion and Democracy
\subsection{Week 1}
\reading[Chapters 7--9 from]{Dahl1989}
\reading[Chapters 2--3 from ]{Herbst1995}
\reading[Chapters 1--3 from ]{Downs1957}
\reading{Mansbridge2003}
\reading[Chapter 1 from]{Riker1988}

\seealso
\reading{Lippman1922}
\reading{Lippman1928}
\reading{Dahl2006}
\reading{Pitkin1967}
\reading{Disch2011}
\reading{HibbingTheiss-Morse2002}
% 2015 article critical of empirical study of representation
% Druckman PolComm comment
\reading{JacobsPage2005}
\reading{BachrachBaratz1962}
\reading{Fishkin2006}
\reading[Chapter 8 from]{Schattschneider1975}
\reading{McCloskyHoffmannOhara1960}
\reading{Erikson1978}

% - Week 2 (Jan. 22): What are opinions?
\subsection{Week 2}
\reading{EaglyChaiken1998}
\reading[Chapter 3 from ]{Zaller1992}

\seealso
\reading{DruckmanLupia2000}


% - Week 3 (Jan. 29): Survey and questionnaire methods
\subsection{Week 3}

\reading{TourangeauRasinski1988}
\reading{BishopTuchfarberOldendick1986}
\reading{SchuldtKonrathSchwarz2011}
\reading{ZallerFeldman1992}
% basic reading on sampling?

\seealso
\reading{Grovesetal2009}
\reading{Lohr2010}
\reading{Berinskyetal2011}
\reading{SchumanPresser}
\reading[]{SchaefferPresser2003}
\reading[]{KrosnickJuddWittenbrink2005}
\reading[]{RevillaSarisKrosnick2013}

\reading{YeagerLarsonKrosnickTompson2010}
\reading{Krosnicketal2002}
\reading{Brady2000}

% - Week 4 (Feb. 5): Opinion dynamics and change (and macro opinion analysis)
\subsection{Week 4}

\reading{DruckmanLeeper2012b}
\reading[Chapter 10 from]{PageShapiro1992}
% something from Macro Polity
% basic reading on time-series analysis

\seealso
\reading{Callegaroetal2015}
\reading{AndresGolschSchmidt}
\reading[]{DruckmanLeeper2012a}
\reading{Gilens2001}
\reading{MacKuenEriksonStimson1989}
\reading{AnsolabehereRoddenSnyder2008}
\reading{Sanders2006}
\reading{JohnstonBrady2002}

% - Week 5 (Feb. 12): Information and opinion responsiveness (and experimental methods)
\subsection{Week 5}

\reading{ChongDruckman2007a}
\reading{Kuklinskietal2000}
\citet[90-92]{DelliCarpiniKeeter1997}
\reading[]{DruckmanNelson2003}
% basic reading on experimental design (maybe from handbook?)

\seealso
\reading{GainesKuklinskiQuirk2007}
\reading{Holland1986}
\reading{LodgeMcGraw1995}
\reading{LodgeSteenbergenBrau1995}
\reading{Iyengar1990}
\reading{LauRedlawsk2001}
\reading{GronlundMilner2006}
\reading{TilleyWlezien2008}
\reading{MutzImpersonalInfluence}
\reading{HuckfeldtSprague}
\reading{Nickerson2008}
\reading{BennettIyengar2008}
\reading[]{ChongDruckman2007b}
\reading[]{DruckmanFeinLeeper2012}

\section{Week 6: Reading Week (Feb. 19) - no class meeting}

There will be no class meeting on February 19 (LT Week 6) due to Reading Week. By this point in the course, students should have an idea in mind for their final exam essay topic or even a relatively elaborated proposal for the exam paper. To obtain feedback on these ideas, students will be assigned to groups of 3-4 students. Groups will meet during Reading Week (at a time convenient for all involved) and provide peer feedback on these formulations. Written proposals should be distributed to group members in advance of the meetings.


% - Week 7 (Feb. 26): Predispositions and Constraint
\subsection{Week 7}
\reading{JostFedericoNapier2009}
\reading{Feldman1988}
% genetics reading
\reading{Gerberetal2011}

\seealso
\reading{BrewerGross2005}
\reading{Converse1964}
\reading{JenningsNiemi1975}
\reading{Lacy2001a}
\reading{CharneyEnglish2012}

% - Week 8 (Mar. 4): Motivated reasoning
\subsection{Week 8}

\reading{TaberLodge2006}
\reading[]{DruckmanBolsen2011}
\reading[]{SlothuusdeVreese2010}
\reading[]{LeeperSlothuus2014}
\reading[]{Evans2008}

\seealso
\reading{Kunda1990}
\reading[]{MoldenHiggins2004}
\reading[]{RedlawskCivettiniEmmerson2010}
\reading[]{DruckmanPetersonSlothuus2013}


% - Week 9 (Mar. 11): Heuristics and cognitive biases
\subsection{Week 9}

\reading[]{Chaiken1980}
\reading[]{MarksMiller1987}
\reading{Petersenetal2011}
\reading{Healy2010}
\reading[]{Petersenetal2011}

\seealso
\reading[]{LauRedlawsk2001}
\reading[]{DanceySheagley2012}
\reading[]{TverskyKahneman1974}
\reading[]{KuklinskiHurley1994}
\reading[]{Hobolt2007}
\reading[]{Arceneaux2007}
\reading[]{Valentinoetal2002}
\reading[]{Lupu2013}

% - Week 10 (Mar. 18): Vote choice and behavior?
%% TOULOUSE CONFERENCE; NEED TO RESCHEDULE
\subsection{Week 10}

\seealso

% - Week 11 (Mar. 25): Revision sessions
% CLASS IS CANCELLED DUE TO EASTER HOLIDAY; NEED TO RESCHEDULE
\subsection{Week 11}

\seealso



% load bibtext, but don't generate a bibliography
\bibliographystyle{plain}
\nobibliography{References}

\end{document}