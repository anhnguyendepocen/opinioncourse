%\documentclass[handout]{beamer} %Makes Handouts
\usetheme{Singapore} %Gray with fade at top
\useoutertheme[subsection=false]{miniframes} %Supppress subsection in header
\useinnertheme{rectangles} %Itemize/Enumerate boxes
\usecolortheme{seagull} %Color theme
\usecolortheme{rose} %Inner color theme

\definecolor{light-gray}{gray}{0.75}
\definecolor{dark-gray}{gray}{0.55}
\setbeamercolor{item}{fg=light-gray}
\setbeamercolor{enumerate item}{fg=dark-gray}

\setbeamertemplate{navigation symbols}{}
\setbeamertemplate{mini frames}[default]
\setbeamercovered{dynamics}
\setbeamerfont*{title}{size=\Large,series=\bfseries}

%\setbeameroption{notes on second screen} %Dual-Screen Notes
%\setbeameroption{show only notes} %Notes Output

\newcommand{\heading}[1]{\noindent \textbf{#1}\\ \vspace{1em}}

\usepackage{bbding,color,multirow,times,ccaption,tabularx,graphicx,verbatim,booktabs,fixltx2e}
\usepackage{colortbl} %Table overlays
\usepackage[english]{babel}
\usepackage[latin1]{inputenc}
\usepackage[T1]{fontenc}

%\author[]{Thomas J. Leeper}
\institute[]{
  \inst{}%
  Department of Political Science and Government\\Aarhus University
}

\usepackage{tikz}
\usetikzlibrary{shapes,arrows}

\title{Methods Supplementary Lecture 1:\\Survey Sampling and Design}

\date[]{}

\begin{document}

\frame{\titlepage}

\frame{\tableofcontents}

\section{Populations}
\frame{\tableofcontents[currentsection]}

\subsection{Representativeness}
\frame{\tableofcontents[currentsubsection, subsectionstyle=show/shaded, sectionstyle=show/hide]}

\frame{
	\frametitle{Inference Population}
	\begin{itemize}\itemsep1em
		\item We want to speak to a population
		\item But what population is it?
		\item<2-> Example: ``The UK population''
	\end{itemize}
}

\frame{
	\frametitle{Population Census}
	\begin{itemize}\itemsep0.5em
		\item All population units are in study
		\item<2-> History of national censuses
			\begin{itemize}
				\item<2-> Denmark 1769--1970 (sporadic)
				\item<2-> U.S. 1790 (decennial)
				\item<2-> India 1871 (decennial)
			\end{itemize}
		\item<3-> Other kinds of census
			\begin{itemize}
				\item<3-> Citizen registry
				\item<3-> Commercial, medical, government records
				\item<3-> ``Big data''
			\end{itemize}
	\end{itemize}
}

\frame{
	\frametitle{Advantages and Disadvantages}
	\begin{itemize}\itemsep1em
		\item Advantages
			\begin{itemize}
				\item<2-> Perfectly representative
				\item<2-> Sample statistics are population parameters
			\end{itemize}
		\item Disadvantages
			\begin{itemize}
				\item<3-> Costs
				\item<3-> Feasibility
				\item<3-> Need
			\end{itemize}		
	\end{itemize}
}


\frame{
	\frametitle{Representativeness}
	\begin{itemize}\itemsep1em
		\item What does it mean for a sample to be representative?
		\item<2-> Different conceptions of representativeness:
			\begin{itemize}
			\item Design-based: A sample is representative because of how it was drawn (e.g., randomly)
			\item Demographic-based: A sample is representative because it resembles in the population in some way (e.g., same proportion of women in sample and population, etc.)
			\item Expert judgement: A sample is representative as judged by an expert who deems it ``fit for purpose''
			\end{itemize}
	\end{itemize}
}


\frame{
	\frametitle{Obtaining Representativeness}
	\begin{itemize}\itemsep1em
		\item<1-> Quota sampling (common prior to the 1940s)
		\item<2-> Simple random sampling
		\item<3-> Advanced survey designs
	\end{itemize}
}

\frame{
	\frametitle{Convenience Samples}
	\begin{itemize}\itemsep1em
		\item<1-> What is a convenience sample?
		\item<2-> Different types:
			\begin{itemize}
				\item<2-> Passive/opt-in
				\item<2-> Sample of convenience (not a sample per se)
				\item<2-> Sample matching
				\item<2-> Online panels
			\end{itemize}
		\item<3-> ``Purposive'' samples (common in qualitative studies)
	\end{itemize}
}

\subsection{Sampling Frames}
\frame{\tableofcontents[currentsubsection, subsectionstyle=show/shaded, sectionstyle=show/hide]}

\frame{
	\frametitle{Sampling Frames}
	\begin{itemize}\itemsep0.25em
		\item Definition: Enumeration (listing) of all units eligible for sample selection
		\item Building a sampling frame
			\begin{itemize}
				\item Combine existing lists
				\item Canvass/enumerate from scratch (e.g., walk around and identify all addresses that people might live in)
			\end{itemize}
		\item There might be multiple frames of the sample population (e.g., telephone list, voter list, residential addresses)
		\item List might be at wrong unit of analysis (e.g., households when we care about individuals)
	\end{itemize}
}

\frame{
	\frametitle{Coverage: A Big Issue}
	\begin{itemize}\itemsep0.5em
		\item Coverage: any mismatch between population and sampling frame
			\begin{itemize}
			\item \textit{Undercoverage}: the sampling frame does not include all eligible members of the population (e.g., not everyone has a telephone, so a telephone list does not include all people)
			\item \textit{Overcoverage}: the sampling frame includes ineligible units (e.g., residents of a country are not necessarily citizens so a list of residents has overcoverage for the population of residents)
			\end{itemize}
		\item Coverage of a frame can change over time (e.g., residential mobility, attrition)
	\end{itemize}
}


\frame{
	\frametitle{Multi-frame Designs}
	\begin{itemize}\itemsep1em
		\item Construct one sample from multiple sampling frames
		\item E.g., ``Dual-frame'' (landline and mobile)
		\item Analytically complicated
			\begin{itemize}
				\item Overlap of frames
				\item Sample probabilities in each frame
			\end{itemize} 
	\end{itemize}
}


\subsection{Sampling without a Frame}
\frame{\tableofcontents[currentsubsection, subsectionstyle=show/shaded, sectionstyle=show/hide]}

\frame{
	\frametitle{Sampling without a Sampling Frame}
	\begin{itemize}\itemsep1em
		\item Sometimes we have a population that can be sampled but not (easily) enumerated in full
		\item<2-> Examples
			\begin{itemize}
				\item<2-> Protest attendees
				\item<3-> Streams (e.g., people buying groceries)
				\item<4-> Points in time
			\end{itemize}
		\item<5-> Population is the sampling frame
	\end{itemize}
}

\frame{
	\frametitle{Rare or ``hidden'' populations}
	\begin{itemize}\itemsep1em
		\item Big concern: coverage!
		\item<2-> Solutions?
			\begin{itemize}
				\item<3-> Snowball sampling
				\item<3-> Informant sampling
				\item<3-> Targeted sampling
				\item<3-> Respondent-driven sampling
			\end{itemize}
	\end{itemize}
}





\section{Parameters and Estimates}
\frame{\tableofcontents[currentsection]}

\frame{
	\frametitle{Inference from Sample to Population}
	\begin{itemize}\itemsep1em
	\item We want to know population parameter $\theta$
	\item We only observe sample estimate $\hat{\theta}$
	\item We have a guess but are also uncertain
	\vspace{1em}
	\item<2-> What range of values for $\theta$ does our $\hat{\theta}$ imply?
	\item<2-> Are values in that range large or meaningful?
	\end{itemize}
}

\frame{
	\frametitle{How Uncertain Are We?}
	\begin{itemize}\itemsep1em
	\item Our uncertainty depends on sampling procedures (we'll discuss different approaches shortly)
	\item Most importantly, \textit{sample size}
		\begin{itemize}
		\item As $n \rightarrow \infty$, uncertainty $\rightarrow 0$
		\end{itemize}
	\item We typically summarize our uncertainty as the \textit{standard error}
	\end{itemize}
}

\frame{
	\frametitle{Standard Errors (SEs)}
	\begin{itemize}\itemsep1em
	\item Definition: ``The standard error of a sample estimate is the average distance that a sample estimate ($\hat{\theta}$) would be from the population parameter ($\theta$) if we drew many separate random samples and applied our estimator to each.''
\end{itemize}
}


\frame{
	\frametitle{What affects size of SEs?}
	\begin{itemize}\itemsep0.5em
	\item Larger variance in $x$ means smaller SEs
	\item More unexplained variance in $y$ means bigger SEs
	\item More observations reduces the numerator, thus smaller SEs
	\item Other factors:
		\begin{itemize}
		\item Homoskedasticity
		\item Clustering
		\end{itemize}
	\item Interpretation:
		\begin{itemize}
		\item Large SE: Uncertain about population effect size
		\item Small SE: Certain about population effect size
		\end{itemize}
	\end{itemize}
}



\frame{
	\frametitle{Ways to Express Our Uncertainty}
	\begin{enumerate}\itemsep1em
	\item Standard Error
	\item Confidence interval
	\item p-value
	\end{enumerate}
}


\frame{
	\frametitle{Confidence Interval (CI)}
	\begin{itemize}\itemsep1em
	\item Definition: Were we to repeat our procedure of sampling, applying our estimator, and calculating a confidence interval \textit{repeatedly} from the population, a fixed percentage of the resulting intervals would include the true population-level slope.
	\item Interpretation: If the confidence interval overlaps zero, we are uncertain if $\beta$ differs from zero
	\end{itemize}
}

% CI PLOT

\frame{
	\frametitle{Confidence Interval (CI)}
	\begin{itemize}\itemsep1em
	\item A CI is simply a range, centered on the slope
	\item Units: Same scale as the coefficient ($\frac{y}{x}$)
	\item We can calculate different CIs of varying \textit{confidence}
		\begin{itemize}
		\item Conventionally, $\alpha = 0.05$, so 95\% of the CIs will include the $\beta$
		\end{itemize}
	\end{itemize}

}





\frame{
	\frametitle{p-value}
	\begin{itemize}\itemsep1em
	\item A summary measure in a hypothesis test
	\item General definition: ``the probability of a statistic as extreme as the one we observed, if the null hypothesis was true, the statistic is distributed as we assume, and the data are as variable as observed''
	\item Definition in the context of a mean: ``the probability of a mean as large as the one we observed \dots''
	%\item Procedure: Calculate a $t$-statistic, then go back to the $t$-distribution to determine how likely a given $t$-ratio is
	\end{itemize}
}




\frame{
	\frametitle{The p-value is not:}
	\begin{itemize}\itemsep1em
	\item The probability that a hypothesis is true or false
	\item A reflection of our confidence or certainty about the result
	\item The probability that the true slope is in any particular range of values
	\item A statement about the importance or substantive size of the effect
	\end{itemize}
}


\frame{
	\frametitle{Significance}
	\begin{enumerate}\itemsep1em
    	\item Substantive significance
    		\begin{itemize}
        		\item<2-> Is the effect size (or range of possible effect sizes) \textit{important} in the real world?
    		\end{itemize}
    	\item Statistical significance
    		\begin{itemize}
        		\item<3-> Is the effect size (or range of possible effect sizes) larger than a predetermined threshold?
        		\item<3-> Conventionally, $p \le 0.05$
    		\end{itemize}
	\end{enumerate}
}







\section{Simple Random Sampling}
\frame{\tableofcontents[currentsection]}

\frame{
	\frametitle{Simple Random Sampling (SRS)}
	\begin{itemize}\itemsep1em
		\item Advantages
			\begin{itemize}
				\item Simplicity of sampling
				\item Simplicity of analysis
			\end{itemize}
		\item Disadvantages
			\begin{itemize}
				\item Need sampling frame and units without any structure
				\item Possibly expensive
			\end{itemize}
	\end{itemize}
}

\frame{
	\frametitle{Sample Estimates from an SRS}
	\begin{itemize}\itemsep1em
		\item Each unit in frame has equal probability of selection
		\item Sample statistics are unweighted
		\item Sampling variances are easy to calculate
		\item Easy to calculate sample size need for a particular variance
	\end{itemize}
}

\frame{
	\frametitle{Sample mean}
	\begin{equation}
	\bar{y} = \frac{1}{n}\sum_{i=1}^{n}y_i
	\end{equation}
	where $y_i = $ value for a unit, and\\
	$n = $ sample size
	
	\begin{equation}
	SE_{\bar{y}} = \sqrt{(1-f)\frac{s^2}{n}}
	\end{equation}
	where $f = $ proportion of population sampled,\\
	$s^2 = $ sample (element) variance, and\\
	$n = $ sample size
}

\frame{
	\frametitle{Sample proportion}
	\begin{equation}
	\bar{y} = \frac{1}{n}\sum_{i=1}^{n}y_i
	\end{equation}
	where $y_i = $ value for a unit, and\\
	$n = $ sample size
	
	\begin{equation}
	SE_{\bar{y}} = \sqrt{\frac{(1-f)}{(n-1)}p(1-p)}
	\end{equation}
	where $f = $ proportion of population sampled,\\
	$p = $ sample proportion, and\\
	$n = $ sample size
}

\frame{
	\frametitle{Estimating sample size}
	\begin{itemize}\itemsep0.5em
		\item Imagine we want to conduct a political poll
		\item We want to know what percentage of the public will vote for which coalition/party
		\item How big of a sample do we need to make a relatively precise estimate of voter support?
	\end{itemize}
}

\frame{
	\frametitle{Estimating sample size}
	\begin{equation}
		Var(p) = (1-f)\frac{p(1-p)}{n-1}
	\end{equation}
	
	Given the large population:	
	\begin{equation}
		Var(p) = \frac{p(1-p)}{n-1}
	\end{equation}
	
	\vspace{1em}
	Need to solve the above for $n$.
	\begin{equation}
	\only<2->{n = \frac{p(1-p)}{v(p)} = \frac{p(1-p)}{SE^2}}
	\end{equation}
}

\frame{
	\frametitle{Estimating sample size}
    Determining sample size requires:
    	\begin{itemize}
    		\item A possible value of $p$
    		\item A desired precision (SE)
    	\end{itemize}
	\vspace{1em}
	If support for each coalition is evenly matched ($p = 0.5$):
	\begin{equation}
	n = \frac{0.5(1-0.5)}{SE^2} = \frac{0.25}{SE^2}
	\end{equation}
}

\frame{
	\frametitle{Estimating sample size}
	What precision (margin of error) do we want?
	\begin{itemize}
		\item +/- 2 percentage points: $SE = 0.01$
			\begin{equation}
			n = \frac{0.25}{0.01^2} = \frac{0.25}{0.0001} = 2500
			\end{equation}
		\item<2-> +/- 5 percentage points: $SE = 0.025$
			\begin{equation}
			n = \frac{0.25}{0.000625} = 400
			\end{equation}
		\item<3-> +/- 0.5 percentage points: $SE = 0.0025$
			\begin{equation}
			n = \frac{0.25}{0.00000625} = 40,000
			\end{equation}
	\end{itemize}
}

\frame{
	\frametitle{Important considerations}
	\begin{itemize}\itemsep0.5em
		\item Required sample size depends on $p$ and $SE$
		\item<2-> In large populations, population size is irrelevant
		\item<3-> In small populations, precision is influenced by the proportion of population sampled
		\item<4-> In anything other than an SRS, sample size calculation is more difficult
		\item<5-> Much political science research assumes SRS even though a more complex design is actually used
	\end{itemize}
}

\frame{
	\frametitle{Sampling Error}
	\begin{itemize}\itemsep1em
		\item Definition? Reasons why a sample estimate may not match the population parameter
		\item<2-> Unavoidable!
		\item<3-> Sources of sampling error:
			\begin{itemize}
				\item Sampling
				\item Sample size
				\item Unequal probabilities of selection
				\item Non-Stratification
				\item Cluster sampling
			\end{itemize}
	\end{itemize}
}


\section{Complex Survey Design}
\frame{\tableofcontents[currentsection]}

\frame{
	\frametitle{Simple Random Sampling (SRS)}
	\begin{itemize}\itemsep1em
		\item Advantages
			\begin{itemize}
				\item Simplicity of sampling
				\item Simplicity of analysis
			\end{itemize}
		\item Disadvantages
			\begin{itemize}
				\item Need complete sampling frame
				\item Possibly expensive
			\end{itemize}
	\end{itemize}
}

\frame{
	\frametitle{Stratified Sampling}
	\begin{itemize}\itemsep1em
		\item What is it? Random samples within ``strata'' of the population
		\item Why do we do? To reduce uncertainty of our estimates
		\item<2-> Most useful when subpopulations are:
		    \begin{enumerate}
		    \item identifiable in advance
		    \item differ from one another
		    \item have low within-stratum variance
		    \end{enumerate}
	\end{itemize}
}

\frame{
	\frametitle{Stratified Sampling}
	\begin{itemize}\itemsep1em
		\item Advantages
			\begin{itemize}
				\item<2-> Avoid certain kinds of sampling errors
				\item<2-> Representative samples of subpopulations
				\item<2-> Often, lower variances (greater precision of estimates)
			\end{itemize}
		\item<3-> Disadvantages
			\begin{itemize}
				\item<4-> Need complete sampling frame
				\item<4-> Possibly (more) expensive
				\item<4-> No advantage if strata are similar
				\item<4-> Analysis is more potentially more complex than SRS
			\end{itemize}
	\end{itemize}
}

\frame{
	\frametitle{Outline of Process}
	\begin{enumerate}
		\item Identify our population
		\item Construct a sampling frame
		\item Identify variables we already have that are related to our survey variables of interest
		\item Stratify or subset or sampling frame based on these characteristics
		\item Collect an SRS (of some size) within each stratum
		\item Aggregate our results
	\end{enumerate}
}

\frame{
	\frametitle{{\normalsize Estimates from a stratified sample}}
	\begin{itemize}\itemsep0.75em
		\item Within-strata estimates are calculated just like an SRS
		\item Within-strata variances are calculated just like an SRS
		\vspace{1em}
		\item Sample-level estimates are weighted averages of stratum-specific estimates
		\item Sample-level variances are weighted averages of stratum-specific variances
	\end{itemize}
}


\frame{
	\frametitle{Design effect}
	\begin{itemize}\itemsep1em
        \item What is it?
		\item<2-> Ratio of variances in a design against a same-sized SRS
		\item<3-> $d^2 = \frac{Var_{stratified}(y)}{Var_{SRS}(y)}$
		\item<4-> Possible to convert design effect into an \textit{effective sample size}:
		\item<4-> $n_{effective} = \frac{n}{d}$
	\end{itemize}
}


\frame{
	\frametitle{How many strata?}
	\begin{itemize}\itemsep1em
		\item How many strata can we have in a stratified sampling plan?
		\item<2-> As many as we want, up to the limits of sample size
	\end{itemize}
}


\frame{
	\frametitle{How do we allocate sample units to strata?}
	\begin{itemize}\itemsep1em
		\item Proportional allocation
		\item Optimal precision
		\item Allocation based on stratum-specific precision objectives
	\end{itemize}
}


\frame{
	\frametitle{Example Setup}
	\begin{itemize}\itemsep1em
		\item Interested in individual-level rate of crime victimization in some country
		\item We think rates differ among native-born and immigrant populations
		\item Assume immigrants make up 12\% of population
		\item Compare uncertainty from different designs ($n=1000$)
	\end{itemize}
}

\frame{
	\frametitle{SRS}
	\begin{itemize}\itemsep0.5em
		\item Assume equal rates across groups ($p=0.10$)
		\item Overall estimate is just $\frac{Victims}{n}$
		\item $SE(p) = \sqrt{\frac{p(1-p)}{n-1}}$
		\item $SE(p) = \sqrt{\frac{0.09}{999}} = 0.0095$
		\item<2-> SEs for subgroups (native-born and immigrants)?
		\item<3-> What happens if we don't get any immigrants in our sample?
	\end{itemize}
}


\frame{
	\frametitle{Proportionate Allocation I}
	\begin{itemize}\itemsep0.75em
		\item Assume equal rates across groups
		\item Sample 880 native-born and 120 immigrant individuals
		\item $SE(p) = \sqrt{Var(p)}$, where
		    \begin{itemize}\itemsep0.75em
		    \item $Var(p) = \sum_{h=1}^{H}(\frac{N_h}{N})^2 \frac{p_h(1-p_h)}{n_h - 1}$
		    \item $Var(p) = (\frac{0.09}{879})(.88^2) + (\frac{0.09}{119})(.12^2)$
		    \item $SE(p) = 0.0095$
		    \end{itemize}
		\item Design effect: $d^2 = \frac{0.0095^2}{0.0095^2} = 1$
	\end{itemize}
}

\frame{
	\frametitle{Proportionate Allocation I}
	\begin{itemize}\itemsep0.75em
		\item Note that in this design we get different levels of uncertainty for subgroups
		\item $SE(p_{native}) = \sqrt{\frac{p(1-p)}{879}} = \sqrt{\frac{0.09}{879}} = 0.010$
		\item $SE(p_{imm}) = \sqrt{\frac{p(1-p)}{119}} = \sqrt{\frac{0.09}{119}} = 0.028$
	\end{itemize}
}



\frame{
	\frametitle{Proportionate Allocation IIa}
	\begin{itemize}\itemsep1em
		\item Assume different rates across groups (immigrants higher risk)
		\item $p_{native}=0.1$ and $p_{imm}=0.3$ (thus $p_{pop} = 0.124$)
		\item $Var(p) = \sum_{h=1}^{H}(\frac{N_h}{N})^2 \frac{p_h(1-p_h)}{n_h - 1}$
		\item $Var(p) = (\frac{0.09}{879})(.88^2) + \frac{0.21}{119})(.12^2))$
		\item $SE(p) = 0.01022$
	\end{itemize}
}

\frame{
	\frametitle{Proportionate Allocation IIa}
	\begin{itemize}\itemsep1em
        \item $SE(p) = 0.01022$
		\item Compare to SRS:
		    \begin{itemize}
		        \item $SE(p) = \sqrt{\frac{0.124(1-0.124)}{n-1}} = 0.0104$
		    \end{itemize}
		\item Design effect: $d^2 = \frac{0.01022^2}{0.0104^2} = 0.9657$
		\item $n_{effective} = \frac{n}{sqrt(d^2)} = 1017$
	\end{itemize}
}

\frame{
	\frametitle{Proportionate Allocation IIa}
	\begin{itemize}\itemsep1em
        \item Subgroup variances are still different
        \item $SE(p_{native}) = \sqrt{\frac{p(1-p)}{879}} = \sqrt{\frac{.09}{879}} = 0.010$
        \item $SE(p_{imm}) = \sqrt{\frac{p(1-p)}{119}} = sqrt{\frac{.21}{119}} = 0.040$
	\end{itemize}
}


\frame{
	\frametitle{Proportionate Allocation IIb}
	\begin{itemize}\itemsep1em
		\item Assume different rates across groups (immigrants lower risk)
		\item $p_{native}=0.3$ and $p_{imm}=0.1$ (thus $p_{pop} = 0.276$)
		\item $Var(p) = \sum_{h=1}^{H}(\frac{N_h}{N})^2 \frac{p_h(1-p_h)}{n_h - 1}$
		\item $Var(p) = (\frac{0.21}{879})(.88^2) + \frac{0.09}{119})(.12^2))$
		\item $SE(p) = 0.014$
	\end{itemize}
}

\frame{
	\frametitle{Proportionate Allocation IIb}
	\begin{itemize}\itemsep1em
		\item $SE(p) = 0.014$
		\item Compare to SRS:
		    \begin{itemize}
		        \item $SE(p) = \sqrt{\frac{0.276(1-0.276)}{n-1}} = 0.0141$
		    \end{itemize}
		\item Design effect: $d^2 = \frac{0.014^2}{0.0141^2} = 0.9859$
		\item $n_{effective} = \frac{n}{sqrt(d^2)} = 1007$
	\end{itemize}
}

\frame{
	\frametitle{Proportionate Allocation IIa}
	\begin{itemize}\itemsep1em
        \item Subgroup variances are still different
        \item $SE(p_{native}) = \sqrt{\frac{p(1-p)}{879}} = \sqrt{\frac{.21}{879}} = 0.0155$
        \item $SE(p_{imm}) = \sqrt{\frac{p(1-p)}{119}} = sqrt{\frac{.09}{119}} = 0.0275$
	\end{itemize}
}



\frame{
	\frametitle{Proportionate Allocation IIc}
	\begin{itemize}\itemsep0.5em
		\item Look at same design, but a different survey variable (household size)
		\item Assume: $\bar{y}_{native}=4$ and $\bar{Y}_imm=6$ (thus $\bar{Y}_{pop} = 4.24$)
		\item Assume: $Var(Y_{native}) = 1$ and $Var(Y_imm) = 3$ and $Var(Y_{pop}) = 4$
		\item $Var(\bar{y}) = \sum_{h=1}^{H}(\frac{N_h}{N})^2 \frac{s_h^2}{n_h}$
		\item $SE(\bar{y}) = \sqrt{\frac{1^2}{880}(.88^2) + \frac{3^2}{120}(.12^2)} = 0.0443$
	\end{itemize}
}

\frame{
	\frametitle{Proportionate Allocation IIc}
	\begin{itemize}\itemsep1em
		\item $SE(\bar{y}) = 0.0443$
		\item Compare to SRS:
		    \begin{itemize}
		        \item $SE(\bar{y}) = \sqrt{\frac{s^2}{n}} = \sqrt{4/1000} = 0.0632$
		    \end{itemize}
		\item Design effect: $d^2 = \frac{0.0443^2}{0.0632^2} = 0.491$
		\item $n_{effective} = \frac{n}{sqrt(d^2)} = 1427$
		\item<2-> Why is $d^2$ so much larger here?
	\end{itemize}
}



\frame{
	\frametitle{Disproportionate Allocation I}
	\begin{itemize}\itemsep1em
        \item Previous designs obtained different precision for subgroups
		\item Design to obtain stratum-specific precision (e.g., $SE(p_h) = 0.02$)
		\item $n_h = \frac{p(1-p)}{v(p)} = \frac{p(1-p)}{SE^2}$
		\item $n_{native} = \frac{0.09}{0.02^2} = 225$
		\item $n_{imm} = \frac{0.21}{0.02^2} = 525$
		\item $n_{total} = 225 + 525 = 750$
	\end{itemize}
}


\frame{
	\frametitle{Disproportionate Allocation II}
	\begin{itemize}\itemsep1em
		\item Neyman optimal allocation
		\item How does this work?
		    \begin{itemize}
		        \item Allocate cases to strata based on within-strata variance
		        \item Only works for one variable at a time
		        \item Need to know within-strata variance
		    \end{itemize}
	\end{itemize}
}

\frame{
	\frametitle{Disproportionate Allocation II}
	\begin{itemize}\itemsep0.5em
		\item Assume big difference in victimization
		\item $p_{native}=0.01$ and $p_{imm}=0.50$  (thus $p_{pop} = 0.0688$)
		\item Allocate according to: $n_h = n \frac{W_h S_h}{\sum_{h=1}^{H} W_h S_h}$
		\item $\sum_{h=1}^{H} W_h S_h = (0.88 * 0.0099) + (0.12 * 0.25) = 0.0387$
		\item $n_{native} = 1000 \frac{0.0087}{0.0387} = 225$
		\item $n_{imm} = 1000 \frac{0.03}{0.0387} = 775$
	\end{itemize}
}

\frame{
	\frametitle{Disproportionate Allocation II}
	\begin{itemize}\itemsep0.5em
		\item $SE(p_{native}) = \sqrt{\frac{p(1-p)}{225}} = \sqrt{\frac{0.0099}{225}} = 0.00663$
		\item $SE(p_{imm}) = \sqrt{\frac{p(1-p)}{775}} = \sqrt{\frac{.25}{775}} = 0.01796$
		\item $Var(p) = \sum_{h=1}^{H}(\frac{N_h}{N})^2 \frac{p_h(1-p_h)}{n_h - 1}$
		\item $Var(p) = (\frac{0.0099}{225})(.88^2) + (\frac{0.25}{775})(.12^2)$
		\item $SE(p) = 0.00622$
	\end{itemize}
}

\frame{
	\frametitle{Disproportionate Allocation II}
	\begin{itemize}\itemsep0.5em
		\item $SE(p) = 0.00622$
		\item Compare to SRS:
		    \begin{itemize}
		        \item $SE(p) = \sqrt{\frac{0.0688(1-0.0688)}{n-1}} = 0.008$
		    \end{itemize}
        \item Design effect: $d^2 = \frac{0.00622^2}{0.008^2} = 0.6045$
        \item $n_{effective} = \frac{n}{sqrt(d^2)} = 1286$
	\end{itemize}
}

\frame{
	\frametitle{Final Considerations}
	\begin{itemize}\itemsep1em
		\item Reductions in uncertainty come from creating homogeneous groups
		\item Estimates of design effects are variable-specific
		\item Sampling variance calculations do not factor in time, costs, or feasibility
	\end{itemize}
}


\subsection{Cluster Sampling}
\frame{\tableofcontents[currentsubsection]}

\frame{
	\frametitle{Cluster Sampling}
	\begin{itemize}\itemsep1em
		\item What is it?
		\item Why do we do?
		\item<2-> Most useful when:
		    \begin{enumerate}
		    \item Population has a clustered structure
		    \item Unit-level sampling is expensive or not feasible
            \item Clusters are similar
		    \end{enumerate}
	\end{itemize}
}

\frame{
	\frametitle{Cluster Sampling}
	\begin{itemize}\itemsep1em
		\item Advantages
			\begin{itemize}
				\item<2-> Cost savings!
				\item<2-> Capitalize on clustered structure
			\end{itemize}
		\item<3-> Disadvantages
			\begin{itemize}
				\item<4-> Units tend to cluster for complex reasons (self-selection)
				\item<4-> Major increase in uncertainty if clusters differ from each other
				\item<4-> Complex to design (and possibly to administer)
				\item<4-> Analysis is much more complex than SRS or stratified sample
			\end{itemize}
	\end{itemize}
}


\frame{
	\frametitle{Cluster Sampling}
	\begin{itemize}\itemsep1em
		\item Number of stages
			\begin{itemize}
				\item One-stage sampling
				\item Two- or more-stage sampling
			\end{itemize}
		\item Number of clusters
		\item Sample size w/in clusters
		\item Everything depends on variability of clusters
	\end{itemize}
}


\frame{
	\frametitle{Sampling Variance for Cluster Sampling}
	\begin{itemize}\itemsep1em
		\item Sampling variance depends on \textit{between}-cluster variation:\\
		$Var(\bar{y}) = (\frac{1-f}{a})(\frac{1}{a-1})(\sum_{\alpha=1}^{a}(\bar{y}_{\alpha} - \bar{y})^2)$
		\item When \textit{between}-cluster variance is high, \textit{within}-cluster variance is likely to be low
			\begin{itemize}
				\item ``Cluster homogeneity''
			\end{itemize}
	\end{itemize}
}

\frame{
	\frametitle{Design Effect for Cluster Sampling}
	\begin{itemize}\itemsep1em
		\item Cluster samples almost always less \textit{statistically} efficient than SRS
		\item Design Effect depends on cluster homogeneity:
		\begin{itemize}\itemsep1em
			\item $d^2 = \frac{Var_{clustered}(y)}{Var_{SRS}(y)}$
			\item $d^2 = 1 + (n_{cluster}-1)roh$
		\end{itemize}
		\item \textit{roh} (\textit{intraclass correlation coefficient}):
		\begin{itemize}
			\item Proportion of unit-level variance that is between-clusters
			\item Generally positive and small (about 0.00 to 0.10)
		\end{itemize}
	\end{itemize}
}


\subsection{Weights}
\frame{\tableofcontents[currentsection]}

\frame{
	\frametitle{Goal of Survey Research}
	\begin{itemize}\itemsep1em
		\item The goal of survey research is to estimate population-level quantities (e.g., means, proportions, totals)
		\item Samples estimate those quantities with uncertainty (sampling error)
		\item Sample estimates are unbiased if they match population quantities
	\end{itemize}
}


\frame{
	\frametitle{Realities of Survey Research}
	\begin{itemize}\itemsep1em
		\item Sample may not match population for a variety of reasons:
			\begin{itemize}
				\item Due to constraints on design
				\item Due to sampling frame coverage
				\item Due to intentional over/under-sampling
				\item Due to nonresponse
				\item Due to sampling error
			\end{itemize}
		\item<2-> Weighting is never perfect
			\begin{itemize}
				\item Limited to work with observed variables
				\item Rarely have good knowledge of coverage, nonresponse, or sampling error
				\item Weighting can increase sampling variance
			\end{itemize}
	\end{itemize}
}

\frame{
	\frametitle{Three Kinds of Weights}
	\begin{itemize}\itemsep1em
		\item Design Weights
		\item Nonresponse Weights
		\item Post-Stratification Weights
	\end{itemize}
}

\frame{
	\frametitle{Design Weights}
	\begin{itemize}\itemsep1em
		\item Address design-related unequal probability of selection into a sample
		\item Applied to \textit{complex survey designs}:
			\begin{itemize}
				\item Disproportionate allocation stratified sampling
				\item Oversampling of subpopulations
				\item Cluster sampling
				\item Combinations thereof
			\end{itemize}
	\end{itemize}
}

\frame{
	\frametitle{Design Weights: SRS}
	\begin{itemize}\itemsep0.5em
		\item Imagine sampling frame of 100,000 units
		\item Sample size will be 1,000 
		\item What is the probability that a unit in the sampling frame is included in the sample?
		\item<2-> $p = \frac{1000}{100,000} = .01$
		\item<3-> Design weight for all units is $w = 1/p = 100$
		\item<3-> SRS is \textit{self-weighting}
	\end{itemize}
}

\frame{
	\frametitle{{\normalsize Design Weights: Stratified Sample}}
	\begin{itemize}\itemsep0.5em
		\item Imagine sampling frame of 100,000 units
			\begin{itemize}
				\item 90,000 Native-born \& 10,000 Immigrants
			\end{itemize}
		\item Sample size will be 1,000 (proportionate allocation)
			\begin{itemize}
				\item 900 Native-born \& 100 Immigrants
			\end{itemize}
		\item What is the probability that a unit in the sampling frame is included in the sample?
			\begin{itemize}
				\item<2-> $p_{native} = \frac{900}{90,000} = .01$
				\item<2-> $p_{Imm} = \frac{100}{10,000} = .01$
			\end{itemize}
		\item<3-> Design weight for all units is $w = 1/p = 100$
		\item<3-> Proportionate allocation is \textit{self-weighting}
	\end{itemize}
}

\frame{
	\frametitle{{\normalsize Design Weights: Stratified Sample}}
	\begin{itemize}\itemsep0.5em
		\item Imagine sampling frame of 100,000 units
		\begin{itemize}
			\item 90,000 Native-born \& 10,000 Immigrants
		\end{itemize}
		\item Sample size will be 1,000 (disproportionate allocation)
		\begin{itemize}
			\item 500 Native-born \& 500 Immigrants
		\end{itemize}
		\item What is the probability that a unit in the sampling frame is included in the sample?
		\begin{itemize}
			\item<2-> $p_{Native} = \frac{500}{90,000} = .0056$
			\item<2-> $p_{Imm} = \frac{500}{10,000} = .05$
		\end{itemize}
		\item<3-> Design weights differ across units: 
			\begin{itemize}
				\item $w_{Native} = 1/p_{Danish} = 178.57$
				\item $w_{Imm} = 1/p_{Imm} = 20$
			\end{itemize}
		\item<3-> Disproportionate allocation is not \textit{self-weighting}
	\end{itemize}
}

\frame{
	\frametitle{Design Weights: Cluster Sample}
	\begin{itemize}\itemsep0.5em
		\item Imagine sampling frame of 1000 units in 5 clusters of varying sizes
		\item Sample size will be 10 each from 3 clusters
		\item What is the probability that a unit in the sampling frame is included in the sample?
			\begin{itemize}
				\item $p = n_{clusters}/N_{clusters} * 1/n_{cluster} = \frac{3}{5} * 1/n_{cluster}$
			\end{itemize}
		\item<3-> Design weights differ across units: 
		\begin{itemize}
			\item Clusters are equally likely to be sampled
			\item Probability of selection within cluster varies with cluster size
		\end{itemize}
		\item<3-> Cluster sampling is rarely \textit{self-weighting}
	\end{itemize}
}


\frame{
	\frametitle{Nonresponse Weights}
	\begin{itemize}\itemsep1em
		\item Correct for nonresponse
		\item Require knowledge of nonrespondents on variables that have been measured for respondents
		\item Requires data are \textit{missing at random}
		\item Two common methods
			\begin{itemize}
				\item Weighting classes
				\item Propensity score subclassification
			\end{itemize}
	\end{itemize}
}

\frame{
	\frametitle{Nonresponse Weights: Example}
	\begin{itemize}\itemsep1em
		\item Imagine immigrants end up being less likely to respond\footnote{{\scriptsize This refers to a lower RR in this particular survey sample, not in general.}}
			\begin{itemize}
				\item $RR_{Native} = 1.0$
				\item $RR_{Imm} = 0.8$
			\end{itemize}
		\item<2-> Using weighting classes:
			\begin{itemize}
				\item $w_{rr,Native} = 1/1 = 1$
				\item $w_{rr,Imm} = 1/0.8 = 1.25$
			\end{itemize}
		\item<2-> Can generalize to multiple variables and strata
	\end{itemize}
}

\frame{
	\frametitle{Post-Stratification}
	\begin{itemize}\itemsep1em
		\item Correct for nonresponse, coverage errors, and sampling errors
		\item<2-> Reweight sample data to match population distributions
			\begin{itemize}
				\item Divide sample and population into strata
				\item Weight units in each stratum so that the weighted sample stratum contains the same proportion of units as the population stratum does
			\end{itemize}
		\item<3-> There are numerous other related techniques
	\end{itemize}
}

\frame{
	\frametitle{Post-Stratification: Example}
	\begin{itemize}\itemsep1em
		\item Imagine our sample ends up skewed on immigration status and gender relative to the population\\
		\vspace{1em}
		\small
		\begin{tabular}{lrrlr}
			\hline
			Group             & Pop. & Sample & Rep.                &              Weight \\ \hline
			Native, Female    &  .45 &     .5 & \onslide<2->{Over}  & \onslide<3->{0.900} \\
			Native, Male      &  .45 &     .4 & \onslide<2->{Under} & \onslide<4->{1.125} \\
			Immigrant, Female &  .05 &    .07 & \onslide<2->{Over}  & \onslide<5->{0.714} \\
			Immigrant, Male   &  .05 &    .03 & \onslide<2->{Under} & \onslide<6->{1.667}\\  \hline
		\end{tabular}
		\item PS weight is just $w_{ps} = N_l / n_l$
	\end{itemize}
}

\frame{
	\frametitle{Post-Stratification}
	\begin{itemize}\itemsep0.5em
		\item Should only be done after correcting for sampling design
		\item Strata must be large ($n>15$)
		\item Need accurate population-level stratum sizes
		\item Only useful if stratifying variables are related to key constructs of interest		
		\item<2-> This is the basis for inference in non-probability samples
		\begin{itemize}
			\item Probability samples make design-based inferences
			\item Non-probability samples post-stratify to obtain descriptive representativeness
		\end{itemize}
	\end{itemize}
}


\frame{
\frametitle{Weighted Analyses}

\begin{itemize}
\item We can analyze data that \textit{should be} weighted without the weights, but they are no longer mathematically representative of the larger population
\item Using the weights is the way to make population-representative claims from survey data
\item Most statistics can be modified to use weights, e.g.:\\
	\begin{itemize}
	\item Unweighted mean: $\frac{1}{n}\sum_{i=1}^{n} x_i$
	\item Weighted mean: $\frac{1}{n}\sum_{i=1}^{n} x_i * w_i$
	\end{itemize}
\end{itemize}
}



\section{Response Rates}
\frame{\tableofcontents[currentsection]}

\frame{
    \frametitle{Response Rates}
    \begin{itemize}\itemsep1em
        \item Why do we care?
        \item<2-> Survey Error
            \begin{itemize}
                \item<2-> Variance
                \item<2-> Bias
            \end{itemize}
        \item<3-> Sample size calculations (and design effects) are based on completed interviews
        \item<4-> Cost, time, and effort
    \end{itemize}
}

\frame{
    \frametitle{Response Rates}
    \begin{itemize}
        \item Imagine we need $n=1000$
        \item How many attempts to obtain that sample:\\
        \vspace{1em}
        \begin{tabular}{l r}\toprule
        Response Rate & Needed Attempts\\ \midrule
        1.00 & 1000 \\
        0.75 & 1333 \\
        0.50 & 2000 \\
        0.25 & 4000 \\
        0.10 & 10,000 \\ \bottomrule
        \end{tabular}
    \end{itemize}
}


\frame{
	\frametitle{Response Rate}
	\begin{itemize}\itemsep1em
		\item Interviews divided by eligibles
		\item $RR = \frac{I}{E}$
		\item Challenges
    		\begin{itemize}
        		\item Unknown eligibility
        		\item Partial interviews
        		\item Non-probability samples
        		\item Complex survey designs
    		\end{itemize}
        \item Cooperation Rate (I's divided by contacts)
	\end{itemize}
}

\frame{
	\frametitle{Disposition Codes}

	Every attempt to interview someone needs to be categorized into a ``disposition code''. The usual codes fall into four broad categories:

	\begin{itemize}\itemsep1em
		\item Interviews
		\item Refusals
		\item Unknowns
		\item Ineligibles
	\end{itemize}
}

\frame<1>[label=codes]{
	\frametitle{Disposition Codes}
	\begin{itemize}\itemsep1em
		\item Complete Interview (I)
		\item Partial Interview (P)
		\item Non-interviews
    		\begin{itemize}
        		\item Refusal (R)
        		\item Non-contact (NC)
        		\item Other (O)
    		\end{itemize}
		\item<2> Unknowns (U)
		\item<2> Ineligibles
	\end{itemize}
}

\frame{
	\frametitle{What is a refusal?}
	\begin{itemize}\itemsep1em
		\item<1-> How do categorize a respondent as a refusal?
		\item<2-> When can we try to convert an apparent refusal?
	\end{itemize}
}


\frame{
	\frametitle{What is a refusal?}
	\begin{itemize}\itemsep0.5em
		\item<1-> ``I don't want to participate.''
		\item<2-> ``I'm too busy to do this right now.''
		\item<3-> ``What do I get for my time?''
		\item<4-> (Hang-up phone without saying anything.)
		\item<5-> ``Okay, but I only have 5 minutes.''
		\item<6-> ``My husband can do it if you call back.''
		\item<7-> ``How did you get my number?''
		\item<8-> ``Go f' yourself.''
	\end{itemize}
}


\againframe{codes}

\frame{
	\frametitle{Eligibility}
	\begin{itemize}\itemsep1em
		\item<1-> Why would an ineligible unit be in our sample?
		\item<2-> How do we determine ineligibility?
		\item<3-> What do we do with ``unknowns''?
	\end{itemize}
}


\frame{
	\frametitle{Response Rates\footnote{Note: Simplified slightly}}
	Without accounting for eligibility of unknowns:
	\begin{itemize}\itemsep0.5em
		\item $RR1 = \frac{I}{(I + P) + (R + NC) + U}$
		\item $RR2 = \frac{I + P}{(I + P) + (R + NC) + U}$
	\end{itemize}

	\vspace{1em}

	Accounting for eligibility of unknowns:
    \begin{itemize}\itemsep0.5em
		\item $RR3 = \frac{I}{(I + P) + (R + NC) + (e*U)}$
		\item $RR4 = \frac{I + P}{(I + P) + (R + NC) + (e*U)}$
		\vspace{0.5em}
		\item $e$ is estimated Pr(eligible) among unknowns
	\end{itemize}
}

\frame{
	\frametitle{Refusal Rates}
	\begin{itemize}\itemsep1em
		\item Related to response rate
		\item Numerator is refusals
		\item E.g., $REF1 = \frac{R}{(I + P) + (R + NC) + U}$
	\end{itemize}
}

\frame{
	\frametitle{Complex Survey Designs}
	\begin{itemize}\itemsep1em
		\item Stratified Sampling (unequal allocation)
    		\begin{itemize}
        		\item Sums of codes weighted by $\frac{1}{p}$
        		\item $p$ is probability of selection
        		\item May want to report stratum-specific rates
    		\end{itemize}
    	\item Multi-stage sampling (e.g., cluster sampling)
        	\begin{itemize}
            	\item RR is product of cluster cooperation and within-cluster response rate
        	\end{itemize}
	\end{itemize}
}


\frame{
	\frametitle{Internet Surveys}
	\begin{itemize}\itemsep1em
		\item For \textit{probability-based samples}, RR is a product of:
    		\begin{itemize}
        		\item Recruitment Rate (RR for panel enrollment)
        		\item Completion Rate (RR for specific survey)
        		\item Profile Rate (in some cases)
        		\item E.g., if Recruitment Rate is 30\% and Completion Rate is 80\%, $RR = 0.3 * 0.8 =$ 24\%
    		\end{itemize}
    	\item For \textit{non-probability samples}, RR is undefined
        	\begin{itemize}
            	\item No sampling involved (so no denominator)
            	\item If from panel, report Completion Rate
            	\item If fully opt-in, there's nothing you can do
        	\end{itemize}
	\end{itemize}
}



\appendix
\frame{}

\end{document}
