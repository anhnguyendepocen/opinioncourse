\documentclass[12pt]{beamer} %Makes presentation
%\documentclass[handout]{beamer} %Makes Handouts
\usetheme{Singapore} %Gray with fade at top
\useoutertheme[subsection=false]{miniframes} %Supppress subsection in header
\useinnertheme{rectangles} %Itemize/Enumerate boxes
\usecolortheme{seagull} %Color theme
\usecolortheme{rose} %Inner color theme

\definecolor{light-gray}{gray}{0.75}
\definecolor{dark-gray}{gray}{0.55}
\setbeamercolor{item}{fg=light-gray}
\setbeamercolor{enumerate item}{fg=dark-gray}

\setbeamertemplate{navigation symbols}{}
\setbeamertemplate{mini frames}[default]
\setbeamercovered{dynamics}
\setbeamerfont*{title}{size=\Large,series=\bfseries}

%\setbeameroption{notes on second screen} %Dual-Screen Notes
%\setbeameroption{show only notes} %Notes Output

\newcommand{\heading}[1]{\noindent \textbf{#1}\\ \vspace{1em}}

\usepackage{bbding,color,multirow,times,ccaption,tabularx,graphicx,verbatim,booktabs,fixltx2e}
\usepackage{colortbl} %Table overlays
\usepackage[english]{babel}
\usepackage[latin1]{inputenc}
\usepackage[T1]{fontenc}

%\author[]{Thomas J. Leeper}
\institute[]{
  \inst{}%
  Department of Political Science and Government\\Aarhus University
}


\title{Are opinions constrained?}

\date[]{September 25, 2013}

\begin{document}

\frame{\titlepage}

\section{Overview}

\frame{
\frametitle{Constraint}
\begin{itemize}\itemsep1em
\item Opinions that go together should be related/correlated
\item ``Belief systems''
\item Implies stability, coherence, and causal precedence
\end{itemize}
}

\frame{
\frametitle{Does constraint signify meaning?}
\begin{enumerate}\itemsep1em
\item Constraint means opinions are reasoned
\item Constraint means opinions are a symptom
\end{enumerate}
}

\frame{
\frametitle{Types of constraint}
\begin{itemize}\itemsep1em
\item Ideology
\item Values
\item Socialization
\item Biological
	\begin{itemize}
	\item Evolved belief systems
	\item Genetic transmission
	\end{itemize}
\end{itemize}
}


\section{Ideology}

\frame{
\frametitle{Ideology}
\begin{itemize}\itemsep1em
\item What is ideology?
\item Is ideology different from opinions? If so, how?
\item What are the benefits of ideology?
\item What are the downsides of ideology?
\end{itemize}
}

% Jost et al. 



\section{Values}

\frame{
\frametitle{Values}
\begin{itemize}\itemsep1em
\item What are values?
\item Are values different from opinions? If so, how?
\item Where do values come from?
\item Are values universal?
\end{itemize}
}

% Feldman


\section{Biology}


\frame{
\frametitle{Biological bases of opinions}
\begin{itemize}\itemsep1em
\item Response to study of socialization
\item Something more fundamental to opinions
\end{itemize}
}


\frame{
\frametitle{Research approaches}
\begin{itemize}\itemsep1em
\item Evolved patterns of opinions
\item Candidate genes
\item Twin studies
\end{itemize}
}

% Smith et al.

\frame{
\frametitle{Evolutionary approaches}
\begin{itemize}\itemsep1em
\item Genes results from adaptations
\item Much of human history was ``small-group'' scale
\item Genetic fitness as root cause of everything
\item Predictions about general principles of opinion
\end{itemize}
}


\frame{
\frametitle{Candidate genes}
\begin{itemize}\itemsep1em
\item Match genomic tests to survey results
\item Correlate identifiable genes with opinions
\item \href{http://people.duke.edu/~echar/4-gene-predict-everything/FOUR_GENES_PREDICT_EVERYTHING.html}{``Four Genes Predict Everything''}
\end{itemize}
}

\frame{
\frametitle{Twin Studies}
\begin{itemize}\itemsep1em
\item Gather data on twins from registries/surveys
\item Compare MZ to DZ twins
\item Through math, partition variation:
	\begin{itemize}
	\item \textbf{A}dditive genetics (Heritability)
	\item \textbf{C}ommon/shared environment
	\item \textbf{U}nique environment
	\end{itemize}
\item Gene-x-environment interactions
\end{itemize}
}

\frame{
\frametitle{Personality}
\begin{itemize}
\item Big Five personality traits appear heritable
	\begin{itemize}
	\item Openness
	\item Conscientiousness
	\item Extraversion
	\item Agreeableness
	\item Neuroticism
	\end{itemize}
\item Effect of genes may be mediated through personality
\end{itemize}
}


\section{Wrap up}

\frame{
\begin{itemize}\itemsep1em
\item How do we integrate constraint with the processing models from last week?
\item<2-> Does constraint make opinions more or less meaningful?
\item<3-> Does constraint imply anything about `knowing one's own interests'?
\item<4-> Does constraint make representation easier or more difficult? Why?
\end{itemize}
}

\end{document}
