\documentclass[12pt]{beamer} %Makes presentation
%\documentclass[handout]{beamer} %Makes Handouts
\usetheme{Singapore} %Gray with fade at top
\useoutertheme[subsection=false]{miniframes} %Supppress subsection in header
\useinnertheme{rectangles} %Itemize/Enumerate boxes
\usecolortheme{seagull} %Color theme
\usecolortheme{rose} %Inner color theme

\definecolor{light-gray}{gray}{0.75}
\definecolor{dark-gray}{gray}{0.55}
\setbeamercolor{item}{fg=light-gray}
\setbeamercolor{enumerate item}{fg=dark-gray}

\setbeamertemplate{navigation symbols}{}
\setbeamertemplate{mini frames}[default]
\setbeamercovered{dynamics}
\setbeamerfont*{title}{size=\Large,series=\bfseries}

%\setbeameroption{notes on second screen} %Dual-Screen Notes
%\setbeameroption{show only notes} %Notes Output

\newcommand{\heading}[1]{\noindent \textbf{#1}\\ \vspace{1em}}

\usepackage{bbding,color,multirow,times,ccaption,tabularx,graphicx,verbatim,booktabs,fixltx2e}
\usepackage{colortbl} %Table overlays
\usepackage[english]{babel}
\usepackage[latin1]{inputenc}
\usepackage[T1]{fontenc}

%\author[]{Thomas J. Leeper}
\institute[]{
  \inst{}%
  Department of Political Science and Government\\Aarhus University
}


\title{Preview of week 13}

\date[]{November 20, 2013}

\begin{document}

\frame{\titlepage}

\frame{
\heading{Overview for next week}
\begin{itemize}\itemsep2em
\item Skip week 13 readings
\item Start on week 14 readings
\item Discuss the exam
\item Prep for the exam in groups
\end{itemize}
}

\frame{
\frametitle{Exam Preparation}
\begin{itemize}\itemsep1em
\item Form groups of three
\item For next week, write a 500-word essay that addresses the prompt on the next slide (also on course notes)
\item Send essay to your group members and me by Tuesday at 12:00 (noon)
\item Provide feedback to one another on your essays in class
\item I will meet with each group during class
\end{itemize}
}

\frame{
\frametitle{Exam Preparation}

\begin{quote}
Representation is historically defined as a one-way link between public (or public opinions specifically) and elected representatives. Yet much of the public's information about politics comes (directly or indirectly) from elected officials and other politicians. The public's views are thus endogenous to the institutions meant to represent those views. As such, representation as defined by democratic theory is impossible.\\
\end{quote}
Defend, challenge, or qualify the above statement, with reference to literature from the course.
}


\end{document}
