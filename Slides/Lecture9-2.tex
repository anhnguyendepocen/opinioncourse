\documentclass[12pt]{beamer} %Makes presentation
%\documentclass[handout]{beamer} %Makes Handouts
\usetheme{Singapore} %Gray with fade at top
\useoutertheme[subsection=false]{miniframes} %Supppress subsection in header
\useinnertheme{rectangles} %Itemize/Enumerate boxes
\usecolortheme{seagull} %Color theme
\usecolortheme{rose} %Inner color theme

\definecolor{light-gray}{gray}{0.75}
\definecolor{dark-gray}{gray}{0.55}
\setbeamercolor{item}{fg=light-gray}
\setbeamercolor{enumerate item}{fg=dark-gray}

\setbeamertemplate{navigation symbols}{}
\setbeamertemplate{mini frames}[default]
\setbeamercovered{dynamics}
\setbeamerfont*{title}{size=\Large,series=\bfseries}

%\setbeameroption{notes on second screen} %Dual-Screen Notes
%\setbeameroption{show only notes} %Notes Output

\newcommand{\heading}[1]{\noindent \textbf{#1}\\ \vspace{1em}}

\usepackage{bbding,color,multirow,times,ccaption,tabularx,graphicx,verbatim,booktabs,fixltx2e}
\usepackage{colortbl} %Table overlays
\usepackage[english]{babel}
\usepackage[latin1]{inputenc}
\usepackage[T1]{fontenc}

%\author[]{Thomas J. Leeper}
\institute[]{
  \inst{}%
  Department of Political Science and Government\\Aarhus University
}


\title{Preview of\\``Do campaigns help citizens?''}

\date[]{October 30, 2013}

\begin{document}

\frame{\titlepage}

\frame{
\heading{Overview for next week}
}


\frame<4>[label=readings]{
\heading{Readings for Next Week}
\begin{itemize}\itemsep1em
\item Lau and Redlawsk -- ``Voting Correctly?''
\item<2-> Hobolt -- ``Taking Cues on Europe?''
\item<3-> Sniderman -- ``Taking Sides''
\item<4-> Disch -- ``Toward a Mobilization Conception of Democratic Representation''
\end{itemize}
}

\againframe<1>{readings}

\frame{
\heading{Lau and Redlawsk}
\begin{itemize}\itemsep2em
\item How do they define ``correct voting''?
\item Can we actually say whether opinions are correct or incorrect?
\end{itemize}
}

\againframe<2>{readings}

\frame{
\heading{Hobolt}
\begin{itemize}\itemsep1em
\item Why do people vote the way they do on the EU?
\item What do people know and how does it affect their opinions?
\item Can we judge competence based on citizens' knowledge?
\end{itemize}
}

\againframe<3>{readings}

\frame{
\heading{Sniderman}
\begin{itemize}\itemsep2em
\item Does politics allow people to govern?
\item How can people express their opinions in politics?
\item What implications does this have for ideas of citizen rationality?
\end{itemize}
}

\againframe<4>{readings}

\frame{
\heading{Disch}
\begin{itemize}\itemsep2em
\item Most democratic theory (think Dahl) assumes or requires that government follow the people. What are the implications of people following the government?
\item What do opinions mean in this kind of politics?
\item Representation through ``mobilization.'' What does she mean by this? And is it good for citizens, and for democracy?
\end{itemize}
}


\frame{
\heading{Who will write for next week?}
}

\appendix
\frame{}

\end{document}
