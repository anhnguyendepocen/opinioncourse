\documentclass[12pt]{beamer} %Makes presentation
%\documentclass[handout]{beamer} %Makes Handouts
\usetheme{Singapore} %Gray with fade at top
\useoutertheme[subsection=false]{miniframes} %Supppress subsection in header
\useinnertheme{rectangles} %Itemize/Enumerate boxes
\usecolortheme{seagull} %Color theme
\usecolortheme{rose} %Inner color theme

\definecolor{light-gray}{gray}{0.75}
\definecolor{dark-gray}{gray}{0.55}
\setbeamercolor{item}{fg=light-gray}
\setbeamercolor{enumerate item}{fg=dark-gray}

\setbeamertemplate{navigation symbols}{}
\setbeamertemplate{mini frames}[default]
\setbeamercovered{dynamics}
\setbeamerfont*{title}{size=\Large,series=\bfseries}

%\setbeameroption{notes on second screen} %Dual-Screen Notes
%\setbeameroption{show only notes} %Notes Output

\newcommand{\heading}[1]{\noindent \textbf{#1}\\ \vspace{1em}}

\usepackage{bbding,color,multirow,times,ccaption,tabularx,graphicx,verbatim,booktabs,fixltx2e}
\usepackage{colortbl} %Table overlays
\usepackage[english]{babel}
\usepackage[latin1]{inputenc}
\usepackage[T1]{fontenc}

%\author[]{Thomas J. Leeper}
\institute[]{
  \inst{}%
  Department of Political Science and Government\\Aarhus University
}


\title{Preview of\\``Theory and Practice?''}

\date[]{November 27, 2013}

\begin{document}

\frame{\titlepage}

\frame{
\heading{Overview for next week}
}


\frame<4>[label=readings]{
\heading{Readings for Next Week}
\begin{itemize}\itemsep1em
\item<1-> Riker -- {\em Liberalism Against Populism}
\item<2-> Hibbing and Theiss-Morse -- {\em Stealth Democracy}
\item<3-> Sniderman et al. -- {\em Paradoxes of Liberal Democracy}
\end{itemize}
}

\againframe<1>{readings}

\frame{
\heading{Riker}
\begin{itemize}\itemsep1em
\item Is direct democracy a viable alternative to representative democracy?
\item According to Riker, what role is their for public opinion in democracy?
\end{itemize}
}

\againframe<2>{readings}

\frame{
\heading{Hibbing and Theiss-Morse}
\begin{itemize}\itemsep1em
\item What is deliberative democracy? Why would we want?
\item What do people (at least Americans) want out of government?
\item How well do their results generalize?
\end{itemize}
}


\againframe<3>{readings}

\frame{
\heading{Sniderman et al.}
\begin{itemize}\itemsep1em
\item Note: You can find this on the website
\item What does the Danish cartoon crisis say about public opinion?
\end{itemize}
}

\frame{
\heading{Optional readings}
\begin{itemize}\itemsep1em
\item Bachrach and Baratz
\item Sullivan et al.
\item Huddy et al.
\item Fishkin
\item Schattschneider
\item Urbinati and Warren
\end{itemize}
}


\frame{
\heading{Exam preparation}
\begin{itemize}
\item Form groups of three
\item Write a 500-word essay (question on next slide)
\item Send the essay to your group members before class
\item In class, discuss your essays and provide feedback to each other
\item I will meet with each group during class
\end{itemize}
}
\frame{
\heading{Exam preparation}
Respond to the following:
\begin{quote}
The readings from the first week of class offered optimistic ideas about the role of citizens and the representation of their opinions in democracies. This week's readings were more skeptical about the capacity and motivation of citizens and the opportunities (and normative desire) for their opinions to shape government action. Given what you've learned from the course as a whole, how can we reconcile these divergent views? And what theories and evidence can we use toward that goal?
\end{quote}
}


\appendix
\frame{}

\end{document}
